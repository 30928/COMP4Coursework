\chapter{User Manual}

\section{Introduction}

In this manual I will go through the correct way to install and utilise the program to its best abilities. The user will require a computer which meets certain specifications and a CD-ROM holding a zip file of the application. The intended audience for the system was the director of Perfect Publishers Ltd.

\section{Installation}

\subsection{Prerequisite Installation}

As the system has been compiled to a windows executable, the system does not require any programs to be installed on the computer which it is to be used on apart from the system itself. The program was intended to run on Windows 7/8, as these were the operating systems that it was designed on, and the operating system that the client owns.

The following is required for the system to reach its full capabilities:

\begin{itemize}
    \item A Keyboard for user inputs
    \item A Mouse for user inputs
    \item A Hard Disk Drive for storage
    \item A Visual Display Unit for outputs generated by the system
    \item Connection the the internet if the user needs an email sent to their email when they've forgotten their password
    \item At least 512 megabytes of main memory to carry out processing
\end{itemize}

%include as many subsubsections as necessary for each piece of required software
%\subsubsection{Installing Python}

%\subsubsection{Installing PyQt}

%\subsubsection{Etc.}

\subsection{System Installation}

The user requires the CD-R that has the necessary files on for installation. 


After inserting the CD-R, the user will need to open 'My Computer'.

\begin{figure}[H]
    \includegraphics[width=\textwidth]{./Manual/Installation/MyComputer.png}
\end{figure}

The user will then need to right click the CD-R,

\subsection{Running the System}

\section{Tutorial}

\subsection{Introduction}

\subsection{Assumptions}

\subsection{Tutorial Questions}

%include as many subsubsections as necessary for each question in your list
\subsubsection{Question 1}

\subsubsection{Question 2}

\subsection{Saving}

\subsection{Limitations}

\section{Error Recovery}

%include as many subsections as necessary for each error
\subsection{Error 1}

\subsection{Error 2}

\section{System Recovery}

\subsection{Backing-up Data}

\subsection{Restoring Data}