%this code creates a centred table with two columns - one 4cm wide, the other 8.5cm wide
%the ampersand & symbol is used to separate columns
%\\ indicates the end of the row
%\hline puts a line under the previous row

\begin{center}
    \begin{tabular}{|p{4cm}|p{8.5cm}|}
        \hline
        \textbf{Purpose} & \textbf{Example} \\ \hline
        Create a function that returns a result  &
        {\begin{python}[frame=none,numbers=none]
def get_formatted_name(first, last, gender):
    if gender == "F":
        name = "Ms. {0} {1}".format(first,last)
    elif gender == "M":
        name = "Mr. {0} {1}".format(first,last)
    else:
        name = "{0} {1}".format(first,last)
    return name
        \end{python}}
        \\
        \hline
    \end{tabular}
\end{center}

%this code creates a three column table with the text aligned to the left of each column
\begin{center}
\begin{tabular}{|l|l|l|}
    \hline
    \textbf{} & \textbf{} & \textbf{} \\ \hline
    \pythoninline{} & \pythoninline{} & \pythoninline{} \\ \hline
    \pythoninline{} & \pythoninline} & \pythoninline{[} \\ \hline
    \pythoninline{} & \pythoninline{} & \pythoninline{} \\ \hline
    \pythoninline{} & \pythoninline{} & \pythoninline{} \\
    \hline
\end{tabular}
\label{tab:range_examples}
\end{center}
